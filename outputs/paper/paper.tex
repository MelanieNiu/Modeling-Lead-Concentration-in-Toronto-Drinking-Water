% Options for packages loaded elsewhere
\PassOptionsToPackage{unicode}{hyperref}
\PassOptionsToPackage{hyphens}{url}
\PassOptionsToPackage{dvipsnames,svgnames,x11names}{xcolor}
%
\documentclass[
  letterpaper,
  DIV=11,
  numbers=noendperiod]{scrartcl}

\usepackage{amsmath,amssymb}
\usepackage{iftex}
\ifPDFTeX
  \usepackage[T1]{fontenc}
  \usepackage[utf8]{inputenc}
  \usepackage{textcomp} % provide euro and other symbols
\else % if luatex or xetex
  \usepackage{unicode-math}
  \defaultfontfeatures{Scale=MatchLowercase}
  \defaultfontfeatures[\rmfamily]{Ligatures=TeX,Scale=1}
\fi
\usepackage{lmodern}
\ifPDFTeX\else  
    % xetex/luatex font selection
\fi
% Use upquote if available, for straight quotes in verbatim environments
\IfFileExists{upquote.sty}{\usepackage{upquote}}{}
\IfFileExists{microtype.sty}{% use microtype if available
  \usepackage[]{microtype}
  \UseMicrotypeSet[protrusion]{basicmath} % disable protrusion for tt fonts
}{}
\makeatletter
\@ifundefined{KOMAClassName}{% if non-KOMA class
  \IfFileExists{parskip.sty}{%
    \usepackage{parskip}
  }{% else
    \setlength{\parindent}{0pt}
    \setlength{\parskip}{6pt plus 2pt minus 1pt}}
}{% if KOMA class
  \KOMAoptions{parskip=half}}
\makeatother
\usepackage{xcolor}
\setlength{\emergencystretch}{3em} % prevent overfull lines
\setcounter{secnumdepth}{5}
% Make \paragraph and \subparagraph free-standing
\ifx\paragraph\undefined\else
  \let\oldparagraph\paragraph
  \renewcommand{\paragraph}[1]{\oldparagraph{#1}\mbox{}}
\fi
\ifx\subparagraph\undefined\else
  \let\oldsubparagraph\subparagraph
  \renewcommand{\subparagraph}[1]{\oldsubparagraph{#1}\mbox{}}
\fi


\providecommand{\tightlist}{%
  \setlength{\itemsep}{0pt}\setlength{\parskip}{0pt}}\usepackage{longtable,booktabs,array}
\usepackage{calc} % for calculating minipage widths
% Correct order of tables after \paragraph or \subparagraph
\usepackage{etoolbox}
\makeatletter
\patchcmd\longtable{\par}{\if@noskipsec\mbox{}\fi\par}{}{}
\makeatother
% Allow footnotes in longtable head/foot
\IfFileExists{footnotehyper.sty}{\usepackage{footnotehyper}}{\usepackage{footnote}}
\makesavenoteenv{longtable}
\usepackage{graphicx}
\makeatletter
\def\maxwidth{\ifdim\Gin@nat@width>\linewidth\linewidth\else\Gin@nat@width\fi}
\def\maxheight{\ifdim\Gin@nat@height>\textheight\textheight\else\Gin@nat@height\fi}
\makeatother
% Scale images if necessary, so that they will not overflow the page
% margins by default, and it is still possible to overwrite the defaults
% using explicit options in \includegraphics[width, height, ...]{}
\setkeys{Gin}{width=\maxwidth,height=\maxheight,keepaspectratio}
% Set default figure placement to htbp
\makeatletter
\def\fps@figure{htbp}
\makeatother
\newlength{\cslhangindent}
\setlength{\cslhangindent}{1.5em}
\newlength{\csllabelwidth}
\setlength{\csllabelwidth}{3em}
\newlength{\cslentryspacingunit} % times entry-spacing
\setlength{\cslentryspacingunit}{\parskip}
\newenvironment{CSLReferences}[2] % #1 hanging-ident, #2 entry spacing
 {% don't indent paragraphs
  \setlength{\parindent}{0pt}
  % turn on hanging indent if param 1 is 1
  \ifodd #1
  \let\oldpar\par
  \def\par{\hangindent=\cslhangindent\oldpar}
  \fi
  % set entry spacing
  \setlength{\parskip}{#2\cslentryspacingunit}
 }%
 {}
\usepackage{calc}
\newcommand{\CSLBlock}[1]{#1\hfill\break}
\newcommand{\CSLLeftMargin}[1]{\parbox[t]{\csllabelwidth}{#1}}
\newcommand{\CSLRightInline}[1]{\parbox[t]{\linewidth - \csllabelwidth}{#1}\break}
\newcommand{\CSLIndent}[1]{\hspace{\cslhangindent}#1}

\KOMAoption{captions}{tableheading}
\makeatletter
\makeatother
\makeatletter
\makeatother
\makeatletter
\@ifpackageloaded{caption}{}{\usepackage{caption}}
\AtBeginDocument{%
\ifdefined\contentsname
  \renewcommand*\contentsname{Table of contents}
\else
  \newcommand\contentsname{Table of contents}
\fi
\ifdefined\listfigurename
  \renewcommand*\listfigurename{List of Figures}
\else
  \newcommand\listfigurename{List of Figures}
\fi
\ifdefined\listtablename
  \renewcommand*\listtablename{List of Tables}
\else
  \newcommand\listtablename{List of Tables}
\fi
\ifdefined\figurename
  \renewcommand*\figurename{Figure}
\else
  \newcommand\figurename{Figure}
\fi
\ifdefined\tablename
  \renewcommand*\tablename{Table}
\else
  \newcommand\tablename{Table}
\fi
}
\@ifpackageloaded{float}{}{\usepackage{float}}
\floatstyle{ruled}
\@ifundefined{c@chapter}{\newfloat{codelisting}{h}{lop}}{\newfloat{codelisting}{h}{lop}[chapter]}
\floatname{codelisting}{Listing}
\newcommand*\listoflistings{\listof{codelisting}{List of Listings}}
\makeatother
\makeatletter
\@ifpackageloaded{caption}{}{\usepackage{caption}}
\@ifpackageloaded{subcaption}{}{\usepackage{subcaption}}
\makeatother
\makeatletter
\@ifpackageloaded{tcolorbox}{}{\usepackage[skins,breakable]{tcolorbox}}
\makeatother
\makeatletter
\@ifundefined{shadecolor}{\definecolor{shadecolor}{rgb}{.97, .97, .97}}
\makeatother
\makeatletter
\makeatother
\makeatletter
\makeatother
\ifLuaTeX
  \usepackage{selnolig}  % disable illegal ligatures
\fi
\IfFileExists{bookmark.sty}{\usepackage{bookmark}}{\usepackage{hyperref}}
\IfFileExists{xurl.sty}{\usepackage{xurl}}{} % add URL line breaks if available
\urlstyle{same} % disable monospaced font for URLs
\hypersetup{
  pdftitle={Investigating the lead concentration in Toronto water pipe},
  pdfauthor={Yuchao Niu},
  colorlinks=true,
  linkcolor={blue},
  filecolor={Maroon},
  citecolor={Blue},
  urlcolor={Blue},
  pdfcreator={LaTeX via pandoc}}

\title{Investigating the lead concentration in Toronto water
pipe\thanks{Code and data are available
at:https://github.com/MelanieNiu/Investigating-the-risk-of-lead-exposure-through-tap-drinking-water-in-Toronto-}}
\author{Yuchao Niu}
\date{March 8, 2024}

\begin{document}
\maketitle
\begin{abstract}
Lead exposure can have severe health implications especially for the
vulnerable groups of children and pregnant and breastfeeding women. This
report investigates the risk of lead exposure through tap drinking water
in the city of Toronto by analyzing an open dataset and found the risk
level to be negligible. The report also found winter months to correlate
with lower likelihood of lead exposure through tap drinking water. The
report supported that the corrosion control plan implemented by the city
is effective.
\end{abstract}
\ifdefined\Shaded\renewenvironment{Shaded}{\begin{tcolorbox}[interior hidden, borderline west={3pt}{0pt}{shadecolor}, frame hidden, boxrule=0pt, breakable, sharp corners, enhanced]}{\end{tcolorbox}}\fi

\hypertarget{introduction}{%
\section{Introduction}\label{introduction}}

Lead is a naturally occurring element present in the earth's crust. Lead
compounds have been widely used in our society including in paint,
pipes, gasoline, batteries, and cosmetics. Lead can also be introduced
into the environment through industrial activities such as through
mining and spark-ignition engine aircraft. Despite some of their
beneficial uses, lead can be potentially toxic to humans and animals at
elevated levels affecting nearly every organ in the body with the
greatest risk to the nervous system. Children are most susceptible to
the effects of lead, where even low levels of lead in the blood can lead
to behavior and learning problems, slowed growth, lower IQ and anemia
(Reference). Lead can also accumulate in our bodies. Pregnant and
breastfeeding women previously exposed to lead may have it accumulated
in their bodies and released into the bloodstream causing harms to their
fetus or infants.

Exposure to lead through tap drinking water is one major cause of lead
poisoning and has led to past public health crisis {[}flinch
citation{]}. Lead was commonly used in plumbing materials such as water
pipes and solder and can leach into the water supply. New regulations
after 1990s have restricted the use of lead in plumbing, however older
buildings still face the risk of lead leaching. The Canadian Drinking
Water Quality Guidelines establish a maximum allowable concentration of
0.005 ppb for total lead in drinking water, measured directly at the
tap.

Toronto Residential Lead Testing Program supported by the city of
Toronto to test lead concentration in tap drinking water provides data
in this aspect. Utilizing this dataset, my report conducted statistical
analysis to gain insight of the risk of lead exposure through Toronto
tap drinking water from 2014 to the present and provide support that the
corrosion control plan initiated by the city since 2014 has effectively
reduced the risk of lead exposure through tap drinking water. Section
2.1 the Toronto Residential Lead Testing Program dataset ``Non Regulated
Lead Sample'' and its measurement is introduced. In Section 2.2 each
variable of the dataset is explained and visualized. In section 3,
seasonal and regional factors to potential impact lead concentration in
tap drinking water are explored. While regional factors show no clear
relationship with lead concentration in tap drinking water in Toronto,
winter months show the lowest number of drinking water samples with lead
concentration exceeding the allowed concentration set out by public
health guidelines.

\hypertarget{sec-data}{%
\section{Data}\label{sec-data}}

\hypertarget{sec-datasource}{%
\subsection{Data Source and Measurement}\label{sec-datasource}}

The data utilized in this report are retrieved from Open Data Toronto
Portal. The dataset is titled ``Non Regulated Lead Sample''. The data
are being collected by the City's Residential Lead Testing Program which
provides free testing to Toronto residents who have concerns about
potential lead contamination in their drinking water. Residents have the
option to collect and return a water sample kit at any of the six
Toronto Public Health facilities. Detailed instructions are provided on
how to procure an accurate water sample. The lead test results are
shared with the requesting residents when available.

One inherent limitation of this dataset is that sampling is conducted
based on requests of residents other than a systematic collection
method. Therefore not all water pipes of the city are sampled
consistently. Since residents who are more concerned about lead toxicity
or residents who live in households installed with older lead water
pipes are more likely to request sampling, a self-selection bias may be
present and hence limit the usefulness of this dataset. In addition, the
water samples are collected by individual. requesting resident. Even
though with detailed instructions are provided, the City's Residential
Lead Testing Program has no complete control over the exact sample
collection method and location. Privacy of the requesting residents is
preserved; no personal identifier or exact location of sampling is
included in the dataset for privacy reasons.

This dataset contains 13000 observations and 5 columns containing
variables of ID, Sample Number, Sample Date, Partial Postal Code of the
sampling location and Lead Amount detected in the drinking water sample
in ppm. The dataset was last updated on January 24, 2024 and has
captured data from 2014 till present.

The data were cleaned, analyzed, and visualized using the open-source
statistical programming language R (R Core Team 2022) Functionalities
from additional packages tidyverse (Wickham et al. 2019), lubridate
(Grolemund and Wickham 2011) and ggplot (\textbf{ggplot?}) were also
utilized.

\hypertarget{results}{%
\section{Results}\label{results}}

\begin{table}

\end{table}

\newpage

\hypertarget{references}{%
\section*{References}\label{references}}
\addcontentsline{toc}{section}{References}

\hypertarget{refs}{}
\begin{CSLReferences}{1}{0}
\leavevmode\vadjust pre{\hypertarget{ref-citelubridate}{}}%
Grolemund, Garrett, and Hadley Wickham. 2011. {``Dates and Times Made
Easy with {lubridate}.''} \emph{Journal of Statistical Software} 40 (3):
1--25. \url{https://www.jstatsoft.org/v40/i03/}.

\leavevmode\vadjust pre{\hypertarget{ref-citeR}{}}%
R Core Team. 2022. \emph{R: A Language and Environment for Statistical
Computing}. Vienna, Austria: R Foundation for Statistical Computing.
\url{https://www.R-project.org/}.

\leavevmode\vadjust pre{\hypertarget{ref-tidyverse}{}}%
Wickham, Hadley, Mara Averick, Jennifer Bryan, Winston Chang, Lucy
D'Agostino McGowan, Romain François, Garrett Grolemund, et al. 2019.
{``Welcome to the {tidyverse}.''} \emph{Journal of Open Source Software}
4 (43): 1686. \url{https://doi.org/10.21105/joss.01686}.

\end{CSLReferences}



\end{document}
